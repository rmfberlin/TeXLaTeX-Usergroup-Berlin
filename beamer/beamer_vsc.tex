% !TEX TS-program = xelatex
\documentclass[t, % 
							% draft,%
							% handout,%
							xcolor=dvipsnames,%
							% compress,
							hyperref={bookmarks,colorlinks},%
							% smaller,% Grundschrift 10pt
							]%
							{beamer}

% \usepackage{pgfpages}
% \pgfpagesuselayout{2 on 1}[a4paper,border shrink=5mm]


\usepackage{xltxtra,% automatically loads the following packages: fontspec, realscripts, metalogo
						microtype}
% \usepackage{expdlist}
%Spracheinstellungen
\usepackage{polyglossia}
\setdefaultlanguage[spelling = new, babelshorthands = true]{german}
\usepackage[right]{eurosym}	% Euro symbol that is constructed according to the official European Commision’s definitions
\usepackage{dirtree}	% nützliches Paket für Verzeichnisbäume
\usepackage{array}	% new implementation of LATEX’s tabular and array environment
\usepackage{ragged2e}	% für bessere Umbrüche mit Silbentrennung bei Flattersatz
\justifying
\RequirePackage{ragged2e} 
\addtobeamertemplate{column begin}{}{\justifying}
\usepackage{graphicx}
\usepackage{animate}
\usepackage{enumitem}	%to ease customizing the three basic list environments: enumerate, itemize and description
\setlist[description]{leftmargin=24pt}	% Einrückung der description-Aufzählung festlegen
% beamer templates, restoring font, color and template of the current beamer theme  gefunden in https://tex.stackexchange.com/questions/24371/does-enumitem-conflict-with-beamer-for-lists
\setlist[itemize]{label=\usebeamerfont*{itemize item}%
  \usebeamercolor[fg]{itemize item}
  \usebeamertemplate{itemize item}}
\usepackage[autostyle=true,german=guillemets]{csquotes}

% TODO: Heinweis von Sven Seite x von Y einfügen
\usepackage{xurl}	%xurl loads package url by default and de nes possible url breaks for all alphanumerical characters and = / . : * - ~ ' "
\usepackage{hyperref}
\hypersetup{
  bookmarksopenlevel=1,
	unicode=true,
	pdfauthor={Rainer-Maria Fritsch},
	pdftitle={TeXLaTeX Usergroup Berlin},
	pdfsubject={Einführung in Visual Studio Code/git},
	breaklinks,hidelinks,
  colorlinks=true,%
  % linkcolor=black,% Print-PDF
  % urlcolor=black,%
  linkcolor=blue,% Online-PDF
  urlcolor=blue,%
  citecolor=orange,
	filecolor=orange,
}
	
% Farbeinstellungen
% s.a. http://texwelt.de/wissen/fragen/5434/wie-kann-ich-in-beamer-beim-colortheme-seahorse-die-farbe-der-rechten-spalte-und-der-zeile-oben-andern
% \colorlet{mystructure}{gray!30!black}
% \usecolortheme[named=mystructure]{structure}
\setbeamercolor{item}{fg=blue}
% Beamer-Theme
% http://www.hartwork.org/beamer-theme-matrix/
\usetheme{Goettingen}
% \usecolortheme{whale}
\usecolortheme{default}



%
% Die Themen für Schriftzeichen bestimmen die Schriftart.
\usefonttheme{professionalfonts}
%
% Innere Themen spezifizieren die inneren Elemente wie Kopf-, Fußzeile, Sidebar usw. einer Folie.
\useinnertheme{rounded}
% Äußere Themen spezifizieren die Grenzen einer Folie und sagen ob und wo die inneren Elemente liegen.
\useoutertheme{infolines}

% Um halbtransparente Overlays auf seinen Folien zu haben, reicht es folgenden Schalter zu setzen:
\setbeamercovered{transparent=10}

%
% Seitenzahlen in Fußzeile einfügen
\setbeamertemplate{footline}[frame number]

% Navigationssymbole unterdrücken:
\setbeamertemplate{navigation symbols}{}

% Metainformationen
% 
% Titel des Vortrages
\title[Kurzer Titel]{Visual Studio Code}
% Untertitel
\subtitle[]{Eine Arbeitsumgebung für \LaTeX}
% Autor festlegen
\author[RMF]{TeXLaTeX Usergroup Berlin\\\small{Rainer-Maria Fritsch}}
% Datum der Präsentation, alternativ kann mittels \date{\today} auch das aktuelle Datum eingetragen werden.
\date[\today]{\today}
%
%%%%% Ende Präambel %%%%%



\begin{document}
	\begin{frame}{}
		\maketitle
		\begin{center} 
			\tiny{Präsentation erstellt mit {\LaTeX}-Beamer} 
		\end{center}
		
	\end{frame}
	
	\begin{frame}{Was Sie erwartet\ldots}
		\setcounter{tocdepth}{1}
		\tableofcontents
	\end{frame}
	
	% Vor jedem Abschnitt automatisch Inhaltsverzeichnis anzeigen:
	\AtBeginSection[]{
		\begin{frame}
		\setcounter{tocdepth}{2}
		\tableofcontents[currentsection]
    \end{frame}
	}
  
  \section{Visual Studio Code}
  \label{sec:VisualStudioCode}
    \subsection{Vorteile}
		\label{sub:Vorteile}
		\begin{frame}<handout:1|beamer:1>[c]
			\frametitle{Vorzüge}
			\begin{itemize}
				\item Für alle Betriebssysteme verfügbar, gleiches \enquote{Look and Feel}
				\item freie und offene Software (Maintainer Microsoft)
				\item viele hervorragende Plugins
				\item große Entwicklergemeinschaft
				\item für viele Programmiersprachen
				\item deutsche Oberfläche, deutsche Rechtschreibkorrektur
				\item verschiedene Arbeitsbereiche für Projekte
				\item git-Integration
			\end{itemize}
		\end{frame}
		% subsection{Vorteile} (end)
		
		\subsection{Einschränkungen}
		\label{sub:Einschraenkungen}
		\begin{frame}
			\frametitle{\enquote{Einschränkungen} als \LaTeX-Editor}
			\begin{itemize}
				\item vorrangig für Entwickler*innen gemacht
				\item Debuggen und Aufgaben
				\item für die  Entwicklung von Webanwendungen 
				\item individuelle Anpassungen gewöhnungsbedürftig
				\item aber letztendlich genial
			\end{itemize}
		\end{frame}
		% subsection{Einschränkungen} (end)

		\subsection{Vorzüge als \LaTeX-Editor}
		\label{sub:VorzuegeLaTeXEditor}
		\begin{frame}
			\frametitle{Vorzüge als \LaTeX-Editor}
			Visual Studio Code mit Plugin LaTeX Workshop (624.200 mal installiert)
			\begin{itemize}
				\item Syntax-Highlighting
				\item Vorschläge für Befehle beim Tippen
				\item Wiederholung von Begriffen, die vorher schon einmal eingegeben wurden
				\item automatisches Einrücken des Quelltextes
				\item erkennt die notwendigen Übersetzungsläufe
				\item Kennt LaTeX-Direktiven
				\item Quelltext und PDF liegen nebeneinader
				\item hin- und herspringen zwischen beiden Dateien per Mausklick
				\item PDF syncronisiert nicht automatisch; Änderungen an der Präambel können an einer bestimmten Textstelle getestet werden  
			\end{itemize}
		\end{frame}
		% subsection{Vorzüge als \LaTeX-Editor}	(end)

		\subsection{Empfehlenswerte Erweiterungen (Plugins)}
		\label{sub:EmpfehlenswerteErweiterungen}
		\begin{frame}
			\frametitle{Empfehlenswerte Erweiterungen}
			\begin{description}
				\item[LaTeX Workshop] Die Erweiterung, die VS Code zur Arbeitsumgebung für \LaTeX\ macht
				\item[Bookmarks] Sprungmarken an einer Zeile anbringen
				\item[Code Spell Checker] Rechtschreibkorrektur auch für Quellcode
				\item[German – Code Spell Checker]Deutsches Wörterbuch dazu
				\item[TODO Highlight] TODO: und FIXME: als Schlüsselwörter für Aufgaben im Quelltext werden farblich markiert und können auch als Liste ausgegeben werden.\footnote{in den Benutzereinstellungen unter \emph{"todohighlight.include": [...,					
					"**/*.tex"
			],} hinzufügen}
			\end{description}
		\end{frame}
		% subsection{Empfehlenswerte Erweiterungen (Plugins)} (end)
	% section{Visual Studio Code} (end)
	
	\section{git – ein Versionskontrollsystem}
	\label{sec:git-vcs}
	\begin{frame}
		\frametitle{git als Versionskontrolle und Backup}
		Was ist das – Versionskontrolle?
	\end{frame}

	\begin{frame}
		\frametitle{Quellen}
		Sehr gute verständliche Anleitung zu git (englisch)\\
		\url{https://www.git-tower.com/learn/git/ebook/en/command-line/introduction\#start}
	\end{frame}
	% section{git – ein Versionskontrollsystem}	(end)
	
	% \section{Layout/Design} % 
% 	\label{sec:layout_design}
%
% 	% section layout_design (end)
		\begin{frame}<handout:0|beamer:1>[c]
		    \frametitle{Visual Studio Code als Arbeitsumgebung \\für \LaTeX}
				% \animategraphics[height=10mm, autoplay, loop]{6}				{imgpdf/animate_}{0}{10}\Large{Vielen Dank für Ihre Aufmerksamkeit!}\\[1.2\baselineskip]
				\centering
				\Huge{Fragen?}
		\end{frame}



\end{document}